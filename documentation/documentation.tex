\documentclass[german]{article}
\usepackage[utf8]{inputenc} % allow utf-8 input
\renewcommand{\familydefault}{\sfdefault}
\usepackage{hyperref}       % hyperlinks
\usepackage{url}            % simple URL typesetting
\usepackage{booktabs}       % professional-quality tables
\usepackage{amsfonts}       % blackboard math symbols
\usepackage{nicefrac}       % compact symbols for 1/2, etc.
\usepackage{microtype}      % microtypography
\usepackage{graphicx}
\usepackage[ngerman]{babel}
\usepackage[margin=2cm]{geometry}
\usepackage{float}
\title{\textbf{Dokumentation~Laborprojekt\\Systemnahe~Programmierung~2}}

\author{
    Adrian~Waldera~(5932553) \\
    Jannik~Peplau~(1995581) \\
    STG-TINF20IN\\
}

\begin{document}

\maketitle

\section{Pinbelegung}

\begin{table}[H]\centering\begin{tabular}{|l|l|}
\hline
Anschluss am Sensor & Pin am Microcontroller\\\hline
VDD & +5.0 V\\\hline
TRIG & ??\\\hline
ECHO & ??\\\hline
VSS & GND\\\hline
\end{tabular}\caption{Pinbelegung Ultraschallsensor}\end{table}\label{tab-pinsus}

\begin{table}[H]\centering\begin{tabular}{|l|l|l|}
\hline
Pin & Anschluss am LCD-Display & Pin am Microcontroller\\\hline
1 & VDD & +3.3 V\\\hline
2 & VSS & GND\\\hline
3 & SDA & \\\hline
4 & SCL & \\\hline
5 & RST & +3.3 V (über 1k$\Omega$)\\\hline
6 & A & +3.3 V\\\hline
7 & K & GND \\\hline
8 & NC & Nicht verbunden\\\hline
\end{tabular}\caption{Pinbelegung Ultraschallsensor}\end{table}\label{tab-pinsus}

\section{Implementierte Funktionen}
Test

\end{document}